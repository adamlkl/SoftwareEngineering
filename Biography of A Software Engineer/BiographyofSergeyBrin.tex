\documentclass[12pt]{article}
\usepackage[utf8]{inputenc}
\usepackage[left=3cm, right=3cm, top=1cm]{geometry}
\usepackage{titling}

\begin{document}
\begingroup
\fontsize{13pt}{12pt}\selectfont
\paragraph{} \hspace{0pt}\\
Title: The Biography of Sergey Brin \\
Author: Leong Kai Ler \\
Student Number: 15334636 \\
Lecturer: Professor Stephen Barrett \\
Course: CS3012 - Software Engineering \\
Date: 22nd October 2018  
\endgroup
\paragraph{} 
Sergey Brin who is famously known as the  co-founder of Google, was born on 21st of August 1973 in Moscow. His father, Michael Brin was a Mathematician who fled from the Soviet Union and emigrated to the United States of America due to hidden Antisemitism and Jewish persecution being heavily practiced in Russia. Michael later became a Mathematics Professor at University of Maryland while his wife became a specialist scientist at the National Aeronautics and Space Administration (NASA). 
\paragraph{}  
At a very young age, Sergey Brin has already displayed his promising talent and vast interests in Mathematics while studying at the Paint Branch Montessori School and  Eleanor Roosevelt High School in Maryland. He was then endorsed by his father to study computer science when he received his first computer Commodore 64 from his father as a birthday present. At that time personal computers were extortionate and therefore not easily accessible or attainable. Brin was also awarded the National Science Foundation Graduate Fellowship, a scholarship that allowed him to cover his tuition fee during his time studying Mathematics at University Maryland while he was 15.
\paragraph{}
After graduating from University of Maryland with Honours, Brin furthered his master studies in Stanford University, Palo Alto, where he would eventually meet Larry Page, the another co-founder of Google. During his time at Stanford, Brin became interested in the Internet and search engine technologies field. He began actively engaging in related sectors by releasing papers on topics such as information extraction from unstructured sources and information retrieval from substantial collections of texts and scientific data. Besides authoring and co-authoring research papers, he was also involved in software development. Some of his notable work encompasses creating an application that converts TeX generated experimental work into HTML format and a web crawl that downloads latest images from Playboy's website and save them as screen savers.  
\paragraph{}
After graduating from University of Maryland with Honours, Brin furthered his master studies in Stanford University, Palo Alto, where he would eventually meet Larry Page, the another co-founder of Google. During his time at Stanford, Brin became interested in the Internet and search engine technologies field. He began actively engaging in related sectors by releasing papers on topics such as information extraction from unstructured sources and information retrieval from substantial collections of texts and scientific data. Besides authoring and co-authoring research papers, he was also involved in software development. Some of his notable work encompasses creating an application that converts TeX generated experimental work into HTML format and a web crawl that downloads latest images from Playboy's website and save them as screen savers.
\paragraph{}
Sergey Brin and Larry Page met during their orientation at Stanford University in March 1995. At that time Brin was keen on developing data-mining systems while on the other hand, Page was exploring a quantitative approach to evaluate the importance of a research paper. The two of them would argue over everything but then soon found their common interests in problems associated with extracting information from enormous data sets. In 1996, They started collaborating in a project to amalgamating the Stanford Digital Library Project into a single, integrated and universal one. 
\paragraph{}
After that project, the pair proceeded into creating their first web search engine called "Backrub", in which they documented their findings in a paper entitled "Anatomy of a Large-Scale Hypertextual Web Search Engine." These findings later became the basis and foundation of their later and successful product, Google, which was named after a mathematical term "Googol," meaning a 1 followed by 100 zeros. The name signified the amount of data they believed to exist in the web.
\paragraph{}
Everything started when Brin and Page deduced that the most useful websites are the ones that are the most popular. By amassing backlink data generated from Backrub and converting them into a tool of measuring the importance of every single web page using their PageRank algorithm, they began realizing that they could build a search engine based on their research and how it would be able to improve searching information on the web. They were confindent that their idea would be far superior compared to other existing search engines since it is a new technology relied on the algorithm to analyze the relevance and liaison between the backlines connecting one webpage to another, and thus, rank a page according to the number of links and their rank.
\paragraph{}
The promising potentials of their idea encouraged them to further their project by converting Page's dorm into a machine laboratory and Brin's one into a programming office. The laboratory was used to assemble spare parts they scavenged from inexpensive computers into a device for connecting their nascent search engine to Stanford's broadband campus network, while the office acted as a programming center for the testing of their new device on the web. The first version of Google went live in August 1996 on the Stanford web. It was hosted under the domain name, google.stanford.edu. Not long after Google made its debut at the university, it became so popular among the students that the pair have to build more severs to accommodate the search requests coming in. Impressed by the popularity and satisfaction from the students, they decided to extend Google's service to users outside the college. 
\paragraph{}
Page's first approach to introduce the company to the public was to find potential buyers to fund the future operations of Google. However, after several failures to sell the product and being rejected by the buyers such as Alta Vista, Excite and Yahoo!, Brin and Page resorted to an alternative strategy to create a business plan and attract investors. Their plan went successfully with Andy Bechtolsheim, the vice-president of CISCO and he offered \$100,000 to them. With the newly outsourced financing, Brin and Page put their doctoral studies on hold and move to Palo Alto to set up their new office. Google was finally registered as Google.com and as a corporation. 

\paragraph{}
It did not took long for Google to be accepted widely by users due to its satisfying query service and search results. However, the essential factor that contributed to their rampant growth in users was the exponential increase in web users, especially after the Dot.com boom in year 2000. That year, in order to survive the onslaught and evisceration of other tech companies, Brin proposed to stay low-profile and focus on improving their search engine, rather than advertising for brands like other huge tech companies like Alta Vista, Lycos and Excite. This allowed Google to gain immense popularity due to the overwhelmed satisfaction from users, which would later propelled their number of users to 1 billion. Despite the successful launch of the product, the company was facing another problem: Financing. As Brin and Page were against the idea of posting advertisements all over the website, they implemented an alternative for it, Adwords, where during search, advertisements are served using links based on the key word entered by the user. This allow Google avoid flooding the web page with ads and also start to generate revenue simultaneously. 

\paragraph{}
As Google launched its IPO(Initial public offering) in 2004, it became raking in profits worth billions of dollars and Brin decided to move the operating site of the business to Mountain View and have it served as the headquarters. To retain the company's value and collaborating environment, the pair would eventually become the Chief Culture Officer at Google for the next 20 years. During their years of service, the company went on to close a \$1 billion dollar advertising deal with America Online, acquiring potential companies like YouTube and DoubleClick and etc. Aside from that, they also brought the company to spotlight by releasing different language versions of Google and expanding their operations internationally, launching Android as an operating systems for PCs and devices and also Chrome as a new adoption for web browsers. Moreover, they company also provides a wide range of services like Gmail, Google Drive, Google+, Google Book and even their own messaging platform Google Hangouts.

\paragraph{}
Together with Larry Page, Sergey Brin has changed and revolutionized the way most people search information from the web. They created a way to index and manifest data stored in the web and thus, making access to them increasingly easier general users. Google also became an inspiration model and a technological platform for software engineers to create and sell their products online. Due to its popularity, Google became more associated with its users' live where most of their daily activities are gradually becoming more connected to it. From searching information online to sending an email and using other extensive functions in Google, it may seemed that it is actually making people lives easier, but if we were to scrutinize from another perspective, Google is actually promoting heavy information exchange through the web, further facilitating communications, connecting users and enhancing the dependencies between humans and technologies.
\bigskip\noindent

\section*{Reference}

\begin{description}
    \item[Santa Claus] https://www.biography.com/people/sergey-brin-12103333 
    \emph{How to Raise Your Reindeer}.
    \item[The Tooth Fairy] https://astrumpeople.com/sergey-brin-biography/ \\
    https://www.investopedia.com/university/sergey-brin-biography/sergey-brin-success-story.asp
\end{description}
\end{document}